% LaTeX Curriculum Vitae Template
%
% Copyright (C) 2004-2008 Jason Blevins <jrblevin@sdf.lonestar.org>
% http://jblevins.org/projects/cv-template
%
% You may use use this document as a template to create your own CV
% and you may redistribute the source code freely. No attribution is
% required in any resulting documents. I do ask that you please leave
% this notice and the above URL in the source code if you choose to
% redistribute this file.

\documentclass[letterpaper]{article}

\usepackage{hyperref}
\usepackage{geometry}
\usepackage[utf8]{inputenc}
\usepackage[norsk]{babel}

% Uncomment the following lines to use the Palatino font.  Remove the
% [osf] bit if you don't like the old style figures.
%
% \usepackage[T1]{fontenc}
% \usepackage[osf]{mathpazo}

% Set your name here
\def\name{Christian Jonassen}

% The following metadata will show up in the PDF properties
\hypersetup{
  colorlinks = true,
  urlcolor = black,
  pdfauthor = {\name},
  pdftitle = {\name: Curriculum Vitae},
  pdfsubject = {Curriculum Vitae},
  pdfpagemode = UseNone
}

\geometry{
  body={6.5in, 8.5in},
  left=1.0in,
  top=1.25in
}

% Customize page headers
\pagestyle{myheadings}
\markright{\name}
\thispagestyle{empty}

% Customize section headings
\usepackage{sectsty}
\sectionfont{\rmfamily\mdseries\Large}
\subsectionfont{\rmfamily\mdseries\scshape\normalsize}

% Alternative section fonts:
%\sectionfont{\rmfamily\bfseries\Large}
%\subsectionfont{\rmfamily\mdseries\itshape\large}

% Don't indent paragraphs.
\setlength\parindent{0em}

% Make lists without bullets
\renewenvironment{itemize}{
  \begin{list}{}{
    \setlength{\leftmargin}{1.5em}
  }
}{
  \end{list}
}

\begin{document}

% Place name at left
% {\huge \name}

% Alternatively, print name centered and bold:
\centerline{\LARGE Curriculum Vitae}

\centerline{}

\centerline{\huge \name}

\vspace{0.25in}

\begin{minipage}[t]{0.5\textwidth}
  Født: 14. november, 1988 \\
  Yrke: Student
\end{minipage}
\begin{minipage}[t]{0.5\textwidth}
  Epost: \href{mailto:chrijon@idi.ntnu.no}{\tt chrijon@idi.ntnu.no} \\
  Mobil: 47 65 81 52
\end{minipage}

\section*{Utdanning}

\begin{itemize}
  \item (2011-2012): Utvekslingsstudent ved UCSD.
  \item (2010-2014): Master i datateknikk ved NTNU (startet egentlig
    på den fem-årige linjen).

  \item 2007-2010: Bachelor i informatikk ved NTNU. Bachelorprosjektet
    var å lage en Git-plugin for Smeedee, et verktøy utviklet av
    Capgemini for å overvåke endringer i et repository. www.smeedee.org
\end{itemize}

\section*{Arbeidserfaring fra NTNU}
\begin{itemize}
\item (Vår 2014): Studentassistent i TDT4100: Objektorientert programering.
\item Høst 2013: Undervisningsassistent i TDT4120: Algoritmer og
  datastrukturer. Holdt øvingsforelesninger og organiserte saltider for
  studentassistentene i faget.
\item Vår 2013: Studentassistent i TDT4100: Objektorientert
  programmering. Kåret til beste Piazza-stud.ass.
\item Høst 2012: Studentassistent in TDT4120: Algoritmer og datastrukturer

\item Høst 2010: Undervisningsassistent in IT110: Informatikk basisfag
  (holdt øvingsforelesninger og var med å lage nye øvinger).
\item Høst 2009: Studentassistent i  IT110: Informatikk basisfag

\item Vår 2011: Undervisningsassistent i TDT4100: Objektorientert programmering
\item Høst 2010: Studentassistent i TDT4100: Objektorientert programmering

\item (2010-): Journalist for readme, avisen for Abakus,
  linjeforeningen for datateknikk og kommunikasjonsteknologi. Stort
  sett skrevet om algoritmer.
\item 2010 og 2011: Arrangør av IDI Open og NM i programmering.
\end{itemize}

\section*{Tidligere erfaringer}
\subsection*{Sommer 2013: KodeWorks}
Jobbet med en modifisert Android-plattform. Lagde en del småverktøy og
apps, men det ble også litt kikking på selve Android-treet.
\subsection*{Sommer 2011: ARM Trondheim}
Vi lagde en grafikkdemo for en Android. Det jeg gjorde, var å:
\begin{itemize}
\item Skrive et augmented reality-aktig kamerasystem.
\item Implementere lyd.
\item Implementere meny.
\item Skrive noen shell-scripts.
\item Noe av logikken for i demoen.
\end{itemize}

\subsection*{2010: Bacheloroppgave for Capgemini}
I bacheloroppgaven var jeg med på å skrive en Git-plugin for Smeedee,
et verktøy utviklet av Capgemini. Smeedee er skrevet i C\# og utviklet
stort sett test-drevet, som vår plugin også ble. Vi brukte
et bibliotek som het GitSharp for å få data fra Git-repoet, et
bibliotek der vi også rapporterte inn en del bugs i løpet av
prosessen. 

\section*{Programmeringsspråk jeg er kjent med}
\begin{itemize}
\item C
\item C++
\item C\#
\item Java
\item Python
\item PHP
\item SQL
\end{itemize}

Jeg har også (ganske begrenset) erfaring med LISP, OCaml, Haskell og 
Prolog. Ellers liker jeg også regular expressions.

% Footer
\begin{center}
\begin{footnotesize}
Sist oppdatert: \today \\
\href{http://folk.ntnu.no/chrijon/cv.pdf}{\tt http://folk.ntnu.no/chrijon/cv.pdf}
\end{footnotesize}
\end{center}
\end{document}
