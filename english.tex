% LaTeX Curriculum Vitae Template
%
% Copyright (C) 2004-2008 Jason Blevins <jrblevin@sdf.lonestar.org>
% http://jblevins.org/projects/cv-template
%
% You may use use this document as a template to create your own CV
% and you may redistribute the source code freely. No attribution is
% required in any resulting documents. I do ask that you please leave
% this notice and the above URL in the source code if you choose to
% redistribute this file.

\documentclass[letterpaper]{article}

\usepackage{hyperref}
\usepackage{geometry}
\usepackage[utf8]{inputenc}
\usepackage[english]{babel}

% Uncomment the following lines to use the Palatino font.  Remove the
% [osf] bit if you don't like the old style figures.
%
% \usepackage[T1]{fontenc}
% \usepackage[osf]{mathpazo}

% Set your name here
\def\name{Christian Jonassen}

% The following metadata will show up in the PDF properties
\hypersetup{
  colorlinks = true,
  urlcolor = black,
  pdfauthor = {\name},
  pdftitle = {\name: Curriculum Vitae},
  pdfsubject = {Curriculum Vitae},
  pdfpagemode = UseNone
}

\geometry{
  body={6.5in, 8.5in},
  left=1.0in,
  top=1.25in
}

% Customize page headers
\pagestyle{myheadings}
\markright{\name}
\thispagestyle{empty}

% Customize section headings
\usepackage{sectsty}
\sectionfont{\rmfamily\mdseries\Large}
\subsectionfont{\rmfamily\mdseries\scshape\normalsize}

% Alternative section fonts:
%\sectionfont{\rmfamily\bfseries\Large}
%\subsectionfont{\rmfamily\mdseries\itshape\large}

% Don't indent paragraphs.
\setlength\parindent{0em}

% Make lists without bullets
\renewenvironment{itemize}{
  \begin{list}{}{
    \setlength{\leftmargin}{1.5em}
  }
}{
  \end{list}
}

\begin{document}

% Place name at left
% {\huge \name}

% Alternatively, print name centered and bold:
\centerline{\LARGE Curriculum Vitae}

\centerline{}

\centerline{\huge \name}

\vspace{0.25in}

\begin{minipage}[t]{0.5\textwidth}
  Born: November 14, 1988 \\
  Occupation: Software Engineer
\end{minipage}
\begin{minipage}[t]{0.5\textwidth}
  Email: \href{mailto:flyrev@gmail.com}{\tt flyrev@gmail.com} \\
  Cell phone: +47 47 65 81 52
\end{minipage}

\section*{Education}

\begin{itemize}
  \item 2010-2014: Masters program in computer science at NTNU - The Norwegian University of Science and Technology. On exchange for one year.
    \item 2011-2012: Exchange student at University of California, San Diego.


  \item 2007-2010: Bachelor in informatics at NTNU. Bachelor thesis
    was a Git plugin for Smeedee, a tool developed by Capgemini to
    monitor changes to a repository. www.smeedee.org
\end{itemize}

\section*{Professional experience}
\subsection*{January 2016 - Present: Software Engineer at Kodeworks}
Technologies:
\begin{itemize}
\item Akka
\item Scala
\item EmberJS
\item Git
\item Linux
\end{itemize}

\subsection*{August 2014 - December 2015: Software Engineer at Telenor Digital}
I Worked a lot on infrastructure, devops and analytics, although there was some core programming. I wrote a lot of unit and integration tests with JUnit and Selenium. During the time I was there I worked on Connect ID, a common login solution for all Telenor Digital services. Most of the things were written in Java, except for the deployment system which was written in Python.

~\\
The services I worked with ran on Amazon Web Services (AWS). For Connect, I made an analytics backend that would parse data from the Woopra analytics service and display these. The front-end was written in Meteor, on which worked on a little bit. During all work there was heavy focus on code review through Gerrit.
~\\

All project management was done with JIRA, however we also experimented with Trello.

~\\
Technologies:
\begin{itemize}
\item Java
\item Python
\item Bash
\item Linux
\item JUnit
\item Selenium
\item Amazon Web Services
\item Meteor (\url{https://www.meteor.com/})
\item Javascript, CSS, HTML
\item Git, Gerrit, GitHub
\item Woopra and Google Analytics
\item Jenkins
\item JIRA
\item Trello
\end{itemize}

\subsection*{Summer 2013: Intern at Kodeworks}
I worked on a project called ``secure mobile platform'' for a small company
called Kodeworks. I worked on both the client and web
server side for communication with a phone, fixed bugs and language
issues, as well as making apps for testing the phone itself. Technologies: Java, Android, Scala.

\subsection*{Summer 2011: Summer intern at ARM Trondheim}
I participated in writing a demo in C++ and OpenGL running on a Mali
graphics processor. The demo ran on the Odroid, which is an Android
platform.
My tasks and responsibilities included:
\begin{itemize}
\item Writing an augmented reality-style camera system that became
  part of the actual demo.
\item Writing a camera system for debugging purposes.
\item Main responsibility for all sound.
\item Touch button menu.
\item Writing occasional shell scripts.
\end{itemize}

\subsection*{2010: Bachelor thesis for Capgemini}
For my bachelor thesis, I was in a group writing a Git plugin for Smeedee, a tool developed by Capgemini to monitor
changes to a repository. Smeedee itself is written in C\#, and the
plugin was developed using test-driven development. We used the
GitSharp library to communicate with Git, discovering bugs and problems
with that library in the process. The code is open source and can be
found on Github.

\section*{Work experience from NTNU}
\begin{itemize}
\item Spring 2014: Student assistant in TDT4100: Object-oriented programming
\item Fall 2013: Teaching assistant in TDT4120: Algorithms and Data
  Structures
\item Spring 2013: Student assistant in TDT4100: Object-oriented programming
\item Fall 2012: Student assistant in TDT4120: Algorithms and Data Structures

\item Fall 2010: Teaching assistant in IT110: Introduction to informatics
\item Fall 2009: Student assistant in  IT110: Introduction to informatics

\item Spring 2011: Teaching assistant in  TDT4100: Object-oriented programming
\item Spring 2010: Student assistant in  TDT4100: Object-oriented programming

\item 2010 and 2011: Programming contest organizer, IDI Open and NCPC.
\end{itemize}

\subsection*{Spare time and other interests}
\begin{itemize}
\item User experience. See my blog at \url{http://www.near-perfection.com/}
\item StackOverflow/StackExchange profile: \url{http://stackexchange.com/users/133473/christian-jonassen}
\end{itemize}

% Footer
\begin{center}
\begin{footnotesize}
Last updated: \today \\
\end{footnotesize}
\end{center}
\end{document}
